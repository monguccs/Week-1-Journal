\documentclass[conference]{IEEEtran}
\IEEEoverridecommandlockouts
% The preceding line is only needed to identify funding in the first footnote. If that is unneeded, please comment it out.
\usepackage{cite}
\usepackage{amsmath,amssymb,amsfonts}
\usepackage{algorithmic}
\usepackage{graphicx}
\usepackage{textcomp}
\usepackage{xcolor}
\usepackage{hyperref} 

\def\BibTeX{{\rm B\kern-.05em{\sc i\kern-.025em b}\kern-.08em
    T\kern-.1667em\lower.7ex\hbox{E}\kern-.125emX}}
\begin{document}

\title{Journal 3, CS6000}


\author{\IEEEauthorblockN{Mong Sim}
\IEEEauthorblockA{\textit{Computer Science Department} \\
\textit{University of Colocrado Colorado Springs}\\
Colorado Springs, USA\\
msim@uccs.edu}

}

\maketitle

\begin{abstract}
A survey paper takes other works and point out their strength and concentration with being animosity about it. It purpose is to provide a summary of many related works in on paper easy for other who is interested in this type of research too quick find out if their idea is novel. Also, is there other space of research that warrant their attention.

\end{abstract}


\section{Survey Paper Information}

A survey paper should consist of the following:

* The solutions being proposed by the authors?
* How are the solutions different from previously proposed solutions?
* How did the author evaluate their solutions?
* What are the drawbacks of the paper?
* What are the insights you learned from this paper that you think can enhance the understanding of the problem/domain?
* What are the techniques presented in this paper that you think can be useful for solving other problems?
* What do you like/dislike about this paper?

These bullet point should sufficient information about a paper.


\end{document}
